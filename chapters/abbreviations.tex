\addchap{\lsAbbreviationsTitle}
\begin{multicols}{2}
\begin{tabbing}
PAM-L2  \=  Perceptual Assimilation Model of L2 Learning\kill
AM   \>   Autosegmental-metrical\\
AOL  \>    Age of learning\\
AP  \>    Accentual phrase\\
BI   \>   Break index\\
BT  \>    Boundary tone\\
CAH  \>    Contrastive Analysis \\ \> Hypothesis\\
CEFR  \>    Common European \\ \> Framework  of Reference for  \\ \>Languages\\
CIA   \>   Contrastive Interlanguage \\ \> Analysis\\
CLI   \>   Cross-linguistic influence\\
DCT  \>    Discourse Completion Task\\
F  \>    Feminine\\
H  \>    High\\
Hi  \>    Initial high tone\\
ip  \>    Intermediate phrase\\
IP   \>   Intonational phrase\\
IPrA  \>     International Prosodic \\ \>Alphabet\\
L   \>   Low\\
L1  \>    First language \\
L2   \>   Second = foreign language\\
L3  \>    Third language\\
LILt  \>    L2 Intonation Learning theory\\
LOR  \>    Length of residence\\
M   \>   Masculine\\
NA  \>    Nuclear accent\\
NC   \>   Nuclear configuration\\
PA   \>   Pitch accent\\
PAI  \>    Initial pitch accent\\
PAM   \>   Medial pitch accent\\
PAM-L2  \>  Perceptual Assimilation \\ \> Model of L2 Learning\\
SLA  \>    Second language acquisition\\
SLM  \>    Speech Learning Model\\
SVO   \>   Subject Verb Object\\
T  \>     Tone\\
TBU  \>    Tone-bearing unit\\
TL  \>    Target language
\end{tabbing}
\end{multicols}
